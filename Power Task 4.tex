\chapter{Presentation and Discussion}
%I am doing Task 4 %

\textbf{The expected effect of distance and orientation and its correlation with the experimental values}
\begin{itemize} 
\item \textbf{Expected effect of distance:} 
\smallskip 
\\ 
The inverse square law is a fundamental principle in physics that describes how the intensity of a point source of radiation decreases with the square of the distance from the source. This law is applicable to various forms of electromagnetic radiation, including light from LEDs.\\
In the case of LEDs, which emit light in the form of electromagnetic waves, the inverse square law can be expressed mathematically as: 

\[I = \frac{P}{4\pi R^2}\]

\item Expected effect of orientation:\\
\[I = P \cdot A\] \[P  = I \cdot A \cdot \cos \theta\] 
where $\theta$ is the angle between the Area vector and the direction of the incident radiation.\\ 
\end{itemize} 

\textbf{An alternative way to estimate the efficiency of the solar cell}\\ A light meter can be used to measure illuminance (in lux) for calculating the efficiency of a solar cell. The following steps are to be followed for the calculations: 
\begin{itemize} 
\item [1.] Utilize a calibrated light meter to quantify the net illuminance at the location of the solar cell. Illuminance, denoted as $E_v$, represents the luminous flux incident per unit area and is measured in lux (lm/m²).
\item [2.] Convert Lux to Incident Solar Power (irradiance) by making use of the conversion factor based on the spectral sensitivity of the solar cell and the Planckian luminosity function (Planck's law of radiation). This involves accounting for the wavelength-dependent response of the solar cell to incident light. Mathematically, irradiance (E) can be calculated as: 
\[E = K \cdot E_v\] where \(K\) is the conversion factor from Planck's law of radiation. 
\item [3.] Using a multi-meter, measure the electrical power output $P_{out}$ of the solar cell. This necessitates recording both voltage (V) and current (I) characteristics and taking the area under the graph.
\item [4.] Determine the solar cell efficiency $\eta$ through the equation: 
\[\eta \ (\%) = \left (\frac{P_{out}}{E}\right) \times 100\] 
\end{itemize} 
This efficiency metric reflects the solar cell's ability to convert incident solar power into usable electrical power. But the spectral match between the incident light and the solar spectrum is crucial, considering the varying responsivity of the solar cell to different wavelengths. Also, factors such as the angle of incidence and temperature impact the overall performance and should be accounted for in a comprehensive analysis.