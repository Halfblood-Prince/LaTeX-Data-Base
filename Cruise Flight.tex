\chapter{Cruise Flight} \label{chap: Cruise Flight} \section{For cruise flight at maximum endurance} \p For a steady and horizontal cruise flight, thrust provided by the propeller is equal to the total drag force acting on the aircraft. So, \eq{ D \ = \ T \ = \ C_D \cdot \frac{1}{2} \cdot \rho \cdot V^2 \cdot S }{eq:Drag} \eq{ P_{prop} \ = \ D \cdot V \ = \ C_D \cdot \frac{1}{2} \cdot \rho \cdot V^3 \cdot S }{eq:P required} \[ V \ = \ \sqrt{\frac{2W}{S \cdot \rho \cdot C_L }} \] \[ P_{prop} \ = \  C_D \cdot \frac{1}{2} \cdot \rho \cdot \left(\frac{2W}{S \cdot \rho \cdot C_L }\right)^{1.5} \cdot S \] \[ P_{prop} = C_D \cdot \frac{1}{2} \cdot \rho \cdot \left(\frac{2g(M_{structure} + M_{camera} + M_{battery} + M_{solar})}{S \cdot \rho \cdot C_L} \right)^{1.5} \cdot S \] \[ M_{structure} \ = \ 0.25 \ kg \] \[ M_{camera} \ = \ 1 \ kg \] \[ C_D \ = \ 0.044039 \] \[ C_L \ = \ 1.223509 \] \[ \rho \ at \ 1500m \ altitude \ = \ 0.01322573918915 \ kg/m^3 \] \[S \ = \ ( b \cdot c ) \ + \ 0.3 \ m^2 \ = \ 1.5 \ m^2\] \[g \ = \ 3.711 \ m/s^2 \] Substituting these values: \[ P_{prop} = 0.044039 \cdot \frac{1}{2} \cdot 0.01322573918915 \cdot \left(\frac{2 \cdot 3.711 \cdot (0.25 + 1 + M_{battery} + M_{solar})}{1.5 \cdot 0.01322573918915 \cdot 1.223509} \right)^{1.5} \cdot 1.5 \] \[ P_{prop} = 0.000436836 \cdot \left(\frac{7.422 \cdot (1.25 + M_{battery} + M_{solar})}{ 0.024272716 } \right)^{1.5} \] \eq{ P_{prop} = 2.336 \cdot (1.25 + M_{battery} + M_{solar})^{1.5} }{eq:p prop} \section{For operating at night:} \p When the aircraft is operating at night, the on-board battery must be able to provide enough power to compensate for the low efficiency of the propeller and it has to support the communication systems. \\ \p So, \[ P_{required | night} = P_{prop} \cdot \frac{1}{\eta_j} \cdot \frac{1}{\eta_{motor}} + P_{communication} \] \[ P_{required | night} = P_{prop} \cdot \frac{100}{87} \cdot \frac{100}{79} + P_{communication} \] \[ P_{required | night} = P_{prop} \cdot \frac{100}{87} \cdot \frac{100}{79} + 5 \quad [W] \] \[ P_{required | night} = 1.454968718 \cdot P_{prop} + 5 \quad [W] \] \[ E_{required | night} = P_{required | night} \times Duration \ of \ night \] Given: \[ Duration \ of \ night \ = \ 13.2 \ hr \] Substituting that in the above equation, \[ E_{required | night} = (1.454968718 \cdot P_{prop} + 5) \times 13.2 \quad [Whr] \] \[ E_{required | night} = 19.206 \cdot P_{prop} + 6 \quad [Whr] \] \p During the night operation, the whole energy is supplied by the on-board battery. So, the energy available in the battery must be equal to that of the energy required for the night operation. Also, it is given that the energy density of the battery is $ 200 \ Wh/kg $. From that, the mass of the battery can be calculated as: \[ Mass \ of \ the \ battery \ = \ \frac{ E_{battery} }{ Energy \ density } \] \[ Mass \ of \ the \ battery \ = \ \frac{ E_{required | night} }{ Energy \ density } \] \[ M_{battery} \ = \ \frac{ 19.206 \cdot P_{prop} + 6 }{ 200 } \] Distributing the denominator and simplyfing further: \eq{ M_{battery} \ = \ 0.096 \cdot P_{prop} + 0.03 }{eq:m bat} \section{For operating during the daytime:} \p During the daytime, the aircraft is powered by the solar cells installed on its wings. The power generated by the solar cell is used to power the propeller, communication system, camera and it is used to recharge the onboard battery. \\ \p So \[ P_{required|day} = P_{prop} \cdot \frac{1}{\eta_j} \cdot \frac{1}{\eta_{motor}} + P_{communication} + P_{camera} + P_{recharge} \] \[ P_{required|day} = P_{prop} \cdot \frac{100}{87} \cdot \frac{100}{79} + 5 + 6.5 + P_{recharge} \quad [W] \] \eq{ P_{required|day} = 1.455 \cdot P_{prop} + 11.5 + P_{recharge} \quad [W] }{eq:P req day} \subsection{Calculating the power needed for recharging} \vspace{0.2cm} \[ Energy \ needed \ for \ recharging \ = \ Enegy \ needed \ for \ night \ operation \] \eq{ P_{recharge} \ \times \ Duration \ of \ recharging \ = \ Enegy \ needed \ for \ night \ operation }{eq:recharge equation} \p For fast charging of the on-board battery, more surface area of solar cell is required, which in turn will increase the overall weight of the aircraft. So, the least area of the solar cell is required when the charging occurs at the slowest rate possible, i.e., the battery takes the entire duration of the day to get recharged. Hence, the duration of recharging will be equal to the duration of the day on Mars, which is $ 11.42 \ hr $. Substituting this in equation~\ref{eq:recharge equation}, \[ P_{recharge} \ \times \ Duration \ of \ daytime \ = \ Power \ needed \ for \ night \ operation \ \times \ Duration \ of \ night \] \[ P_{recharge} \ = \ \frac{ Power \ needed \ for \ night \ operation \ \times \ Duration \ of \ night }{ Duration \ of \ daytime } \] \[ P_{recharge} \ = \ \frac{ \left(P_{prop} \cdot \frac{100}{87} \cdot \frac{100}{79} + 5 \right) \ \times \ 13.2 }{ 11.42 } \quad [W] \] \[ P_{recharge} \ = \ \frac{ 1.455 \cdot P_{prop} + 5 }{ 0.865 } \quad [W] \] Distributing the denominator and simplifying further: \[ P_{recharge} \ = \ 1.682 \cdot P_{prop} + 5.78 \quad [W] \] Substituting this in equation~\ref{eq:P req day} \[ P_{required|day} = 1.455 \cdot P_{prop} + 11.5 + 1.682 \cdot P_{prop} + 5.78 \quad [W] \] \[ P_{required|day} = 3.137 \cdot P_{prop} + 17.28 \quad [W] \] \p During the daytime, the power required is supplied entirely by the solar cell installed on the aircraft wings. So, \[ P_{required|day} \ = \ Power \ output \ of \ solar \ cell \] \subsection{Calculation of the power output of solar cell} \eq{ Power \ output \ of \ solar \ cell \ = \ Incident \ solar \ power \ \times \ \eta_{solar} }{eq:sol} \[ Incident \ solar \ power \ = \ (Solar\ constant)_{mars} \times A_{solar} \times \cos(\theta) \] \eqvar{ a= $ A_{solar} $ is the area of the solar cell, b= $ \theta $ is the inclination of the sun } Given: \[ \theta \ = \ 45 \textdegree \ and \  (Solar\ constant)_{mars} \ = \ 591.98 \ [W/m^2] \] Substituting them in the above equation \[ Incident \ solar \ power \ = \ 591.98 \times A_{solar} \times \cos(45\textdegree) \quad [W] \] \[ Incident \ solar \ power \ = \ 418.593 \times A_{solar} \quad [W] \] So equation~\ref{eq:sol} becomes \[ Power \ output \ of \ solar \ cell \ = \ 418.593 \times A_{solar} \ \times \ \eta_{solar} \] \[ Power \ output \ of \ solar \ cell \ = \ 418.593 \times A_{solar} \ \times \ 51.4\% \] \[ Power \ output \ of \ solar \ cell \ = \ 215.157 \cdot A_{solar} \quad [W] \] \[ P_{required|day} \ = \ 215.157 \cdot A_{solar} \quad [W] \] \[ 3.137 \cdot P_{prop} + 17.28 \ = \ 215.157 \cdot A_{solar} \quad [W] \] \[ A_{solar} \ = \ \frac{ 3.137 \cdot P_{prop} + 17.28 }{ 215.157 } \quad [m^2] \] \[ Mass \ of \ solar \ cell \ = \ A_{solar} \ \times \ Mass \ density \ of \ solar \ cell \] \[ M_{solar} = \ \frac{ 3.137 \cdot P_{prop} + 17.28 }{ 215.157 } \ \times \ 0.32 \quad [kg] \] Distributing the denominator and simplifying further: \[ M_{solar} = \ 0.0047 P_{prop} + 0.0257  \quad [kg] \] \vspace{0.5cm} \p Substituting $ M_{battery} $ and $ M_{solar} $ in equation~\ref{eq:p prop}, \[
P_{prop} = 2.336 \cdot (1.25 + M_{battery} + M_{solar})^{1.5} \] \[ P_{prop} = 2.336 \cdot (1.25 + \ \ 0.096 \cdot P_{prop} + 0.03 \ + \ 0.0047 P_{prop} + 0.0257 )^{1.5} \] \[ P_{prop} = 2.336 \cdot ( 0.1007 \cdot P_{prop} + 1.3057 )^{1.5} \] Solving the above equation: \begin{center} \boxed{P_{\text{prop}} = 6.325 \, \text{[W]}} \end{center} \vspace{1cm} \p \textbf{The power that needs to be delivered during the day (camera operating):} \vspace{0.1cm}\\ It was proved earlier that: \[ P_{required|day} = 3.137 \cdot P_{prop} + 17.28 \quad [W] \]

Substituting the value of $ P_{prop} $ into the equation and solving it, \[ P_{required|day} = 3.137 \cdot 6.325 + 17.28 \quad [W] \] \begin{center} \boxed{P_{\text{{required|day}}} = 37.122 \, \text{[W]}} \end{center} \vspace{1cm} \p \textbf{The power that needs to be delivered during the night (camera not operating):} \vspace{0.1cm}\\ It was proved earlier that: \[ P_{required | night} = 1.454968718 \cdot P_{prop} + 5 \quad [W] \]

Substituting the value of $ P_{prop} $ into the equation and solving it, \[ P_{required | night} = 1.454968718 \cdot 6.325 + 5 \quad [W] \] \begin{center} \boxed{P_{\text{{required|night}}} = 14.207 \, \text{[W]}} \end{center} \vspace{1cm} \p \textbf{The surface area of the solar cell needed:} \vspace{0.1cm}\\ It was proved earlier that: \[ A_{solar} \ = \ \frac{ 3.137 \cdot P_{prop} + 17.28 }{ 215.157 } \quad [m^2] \]

Substituting the value of $ P_{prop} $ into the equation and solving it, \[ A_{solar} \ = \ \frac{ 3.137 \cdot 6.325 + 17.28 }{ 215.157 } \quad [m^2] \] \begin{center} \boxed{A_{\text{{solar}}} = 0.1725 \, \text{[$m^2$]}} \end{center} \vspace{1cm} \p \textbf{The total mass of the solar cell needed:} \vspace{0.1cm}\\ It was proved earlier that: \[ M_{solar} = \ 0.0047 P_{prop} + 0.0257  \quad [kg] \]

Substituting the value of $ P_{prop} $ into the equation and solving it, \[ M_{solar} = \ 0.0047 \cdot 6.325 + 0.0257  \quad [kg] \] \begin{center} \boxed{M_{\text{{solar}}} = 0.0554 \, \text{[kg]}} \end{center} \vspace{1cm} \p \textbf{The total mass of the battery needed:} \vspace{0.1cm}\\ It was proved earlier that: \[ M_{battery} \ = \ \frac{ 19.206 \cdot P_{prop} + 6 }{ 200 } \]

Substituting the value of $ P_{prop} $ into the equation and solving it, \[ M_{battery} \ = \ \frac{ 19.206 \cdot 6.325 + 6 }{ 200 } \] \begin{center} \boxed{M_{\text{{battery}}} = 0.6374 \, \text{[kg]}} \end{center}
