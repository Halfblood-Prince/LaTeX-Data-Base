\chapter{Theoretical Background}

\textbf{Photoelectric Effect \& Einstein's Photoelectric Equation}\smallskip \\
The photoelectric effect is a quantum phenomenon wherein the interaction between electromagnetic radiation, particularly light, and a material results in the emission of photo-electrons. These electrons play a pivotal role in the realm of condensed matter physics, solid-state physics and quantum chemistry.\\
It is stated mathematically as follows:
\[
E_{incident} = \phi + K.E._{electrons}
\]
\hspace{9cm} where $\phi$ is the work function of the material\\
From Max Plank's Quantum Theory:
\[
E \ = \ h \nu \
\]
Therefore the photoelectric equation becomes:
\[
h \nu \ = \ h \nu_0 \ + \ e \cdot V
\]
\begin{itemize}
    \item where $\nu$ is the frequency of the incident radiation
    \item $\nu_0$ is the threshold frequency of the material
    \item h is Planck's constant
    \item e is the charge of electron
    \item V is the stopping potential (voltage generated by the solar cell)
\end{itemize}
This equation proves that the voltage generated by a solar cell is directly proportional to the frequency of the incident radiation.\medskip \\
\textbf{Stefan–Boltzmann Law} \smallskip \\
The Stefan–Boltzmann law is a fundamental principle in radiation physics that quantifies the intensity of thermal radiation emitted by matter. It articulates the relationship between the temperature of an ideal absorber/emitter, often referred to as a black body, and the total energy radiated per unit surface area per unit time.
Mathematically, the Stefan–Boltzmann law is expressed as:
\[
\frac{P}{A} = E A \sigma T^4
\]
This equation highlights that the radiation power emitted by a body is dependent only on the temperature of the body. So, the surface integral of the power received over the entire area of the sphere of illumination is equal to the power emitted by the body. Thus we can also prove the inverse square law of the radiation power emitted by a body.\\
\vfill
\pagebreak
